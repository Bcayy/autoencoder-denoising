\section{Data Description}

The dataset used in this project is the MNIST handwritten digit dataset, which consists of 60,000 training images and 10,000 test images. Each image is a grayscale image with a resolution of $28 \times 28$ pixels, representing digits from 0 to 9.

To simulate real-world noise, we added synthetic Gaussian noise to the dataset. Specifically, zero-mean Gaussian noise with a standard deviation scaled by a factor of 0.5 was applied to each pixel. The resulting noisy images were clipped to maintain pixel values in the range [0, 1].

All images were normalized to float values between 0 and 1 and reshaped to include a channel dimension, resulting in input tensors of shape $(28, 28, 1)$. No further augmentation or transformation was applied.

This noisy-clean pair setup was used to train the denoising autoencoder, where the model learns to reconstruct the original image from its noisy version.